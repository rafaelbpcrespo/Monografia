\chapter{Introdução}

\section{Introdução}
    A Mineração Web é uma metodologia de recuperação de informação, que trata de descobrir padrões interessantes no conteúdo, na estrutura e na utilização dos sites, segundo \cite{Escobar}. Partindo deste princípio, dentro da Mineração Web existem três setores principais de estudo, sendo eles a Mineração Web de Conteúdo, a Mineração Web de Estrutura e a Mineração Web de Uso. Neste trabalho falaremos um pouco sobre todas estas áreas, dando uma maior ênfase a última citada.

    Este segmento de mineração vem crescendo nos últimos tempos, pois a informação se tornou matéria prima de grande valor tanto para as empresas quanto para os usuários. Devido a quantidade de informação, que tende a aumentar significativamente, por causa do desenvolvimento constante da internet e de bases de dados dinâmicas, começaram a surgir maiores recursos para armazenar e processar maiores volumes de dados, já que seria impossível realizar tais tarefas sem o auxílio da tecnologia, e um desses recursos é a Mineração Web.

	Este segmento da mineração vem chamando a atenção principalmente das empresas de e-commerce nos últimos anos, pois tem sido visto como uma grande arma para “conhecer” seu cliente, que neste caso é o usuário que está navegando no site. Essas empresas visam investir em pesquisa e desenvolvimento de métodos de mineração para aplicar essa prática e assim conseguir reverter os dados obtidos ao seu favor, direcionando para uma estratégia de marketing mais eficaz, por exemplo.

	Todas as empresas que vendem produtos ou serviços, dependem de sua fama e de seus clientes, portanto uma boa relação entre a empresa e o cliente é de extrema importância para que este torne-se fiel à determinada empresa, o problema é que com a globalização, que trouxe à tona os serviços remotos, como vendas online, também conhecido como e-commerce, veio junto uma maior dificuldade de entender e “analisar” o comportamento do cliente, saber o que ele procura, qual a abordagem utilizar, o que oferecer, já que agora este não está mais fisicamente presente, ou seja, não há um contato direto entre vendedor e cliente, e é a partir deste problema que a mineração web passa a ser uma ferramente essencial.

\section{Objetivos e justificativas}

    O objetivo do presente trabalho é o estudo da mineração web, mais especificamente focada na área de mineração web de uso, para que sejam aprimorados métodos existentes ou até mesmo criadas novas formas de aproveitamento dos dados que são coletados nesta mineração.

    A partir deste estudo, espera-se que seja possível realizar a aplicação da mineração web de uso em |um site|, para a partir do resultado gerado poder fazer uma reorganização deste site, tornando-o melhor visualmente e também obtendo resultados significativos dos usuários depois que tiver sido feito todo o trabalho de melhoramento.


\section{Metodologia}


\section{Estrutura do trabalho}
    No Capítulo 2 será fundamentado os tipos de mineração web existentes, aprofundando a explanação na mineração web de uso. No capítulo 3 será mostrada a forma como será estruturado o modelo de mineração web de uso.

