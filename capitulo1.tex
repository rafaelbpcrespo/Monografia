\chapter{Introdução}

\section{Introdução}
    A Mineração Web é uma metodologia de recuperação da informação, que usa ferramentas de mineração de dados para extrair informações tanto do conteúdo das páginas da internet e de sua estrutura de relacionamentos (links), quanto dos registros de navegação dos usuário.

	Este segmento da mineração vem chamando a atenção principalmente das empresas de e-commerce nos últimos anos, pois tem sido visto como uma grande arma para “conhecer” seu cliente, que neste caso é o usuário que está navegando no site. Essas empresas visam investir em pesquisa e desenvolvimento de métodos de mineração para aplicar essa prática e assim conseguir reverter os dados obtidos ao seu favor, direcionando para uma estratégia de marketing mais eficaz, por exemplo.

	Todas as empresas que vendem produtos ou serviços, dependem de sua fama e de seus clientes, portanto uma boa relação entre a empresa e o cliente é de extrema importância para que este torne-se fiel à determinada empresa, o problema é que com a globalização, que trouxe à tona os serviços remotos, como vendas online e pelo telefone, veio junto uma maior dificuldade de entender e “analisar” o comportamento do cliente, saber o que ele procura, qual a abordagem utilizar, o que oferecer, já que agora este não está mais fisicamente presente, ou seja, não há um contato direto entre vendedor e cliente, e é a partir deste problema que a mineração web passa a ser uma ferramente essencial.

\section{Objetivos e justificativas}

\begin{itemize}
\item Item 1
\item Item 2
\item Item 3
\end{itemize}


\section{Estrutura do trabalho}

