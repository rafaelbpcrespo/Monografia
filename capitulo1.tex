\chapter{Introdução}

    A Mineração Web é uma metodologia de recuperação de informação, que trata de descobrir padrões interessantes no conteúdo, na estrutura e na utilização dos sites, segundo Jeria\cite{Escobar}. Partindo deste princípio, dentro da Mineração Web existem três setores principais de estudo, sendo eles a Mineração Web de Conteúdo, a Mineração Web de Estrutura e a Mineração Web de Uso.

    Este segmento de mineração vem crescendo nos últimos tempos, pois a informação se tornou matéria prima de grande valor tanto para as empresas quanto para os usuários. Devido a quantidade de informação, que tende a aumentar significativamente, por causa do desenvolvimento constante da internet e de bases de dados dinâmicas, começaram a surgir maiores recursos para armazenar e processar maiores volumes de dados, já que seria impossível realizar tais tarefas sem o auxílio da tecnologia, e um desses recursos é a Mineração Web.

	Este segmento da mineração vem chamando a atenção principalmente das empresas de e-commerce nos últimos anos, pois tem sido visto como uma grande arma para “conhecer” seu cliente, que neste caso é o usuário que está navegando no site. Essas empresas visam investir em pesquisa e desenvolvimento de métodos de mineração para aplicar essa prática e assim conseguir reverter os dados obtidos ao seu favor, direcionando para uma estratégia de marketing mais eficaz, por exemplo.

	Todas as empresas que vendem produtos ou serviços, dependem de sua fama e de seus clientes, portanto uma boa relação entre a empresa e o cliente é de extrema importância para que este torne-se fiel à determinada empresa, o problema é que com a globalização, que trouxe à tona os serviços remotos, como vendas online, também conhecido como e-commerce, veio junto uma maior dificuldade de entender e “analisar” o comportamento do cliente, saber o que ele procura, qual a abordagem utilizar, o que oferecer, já que agora este não está mais fisicamente presente, ou seja, não há um contato direto entre vendedor e cliente, e é a partir deste problema que a mineração web passa a ser uma ferramenta essencial.

	A eficiência de análise dessas informações, chamadas perfis, depende da quantidade e da qualidade das informações registradas como perfis. Com esse enfoque é necessário considerar aspectos de captura dos perfis por onde os usuários das páginas passarem.

    Muitos usuários hoje em dia tem péssimas experiências com usabilidade e navegação por sites, por exemplo se deparam com muita dificuldade em encontrar um produto ou um conteúdo dentro de um site e por isso se irritam e deixam de acessar aquele site. Muitas vezes esses usuários insatisfeitos fazem suas queixas sobre o site em alguma outra página da web que é destinada para isso, ou algum fórum e nenhuma atitude é tomada em relação a isso.

\section{Objetivos e justificativas}

    A partir deste estudo, espera-se que seja possível realizar a aplicação da mineração web de uso em |um site|, para a partir do resultado gerado poder fazer uma reorganização deste site, tornando-o melhor visualmente e também obter resultados significativos dos usuários depois que tiver sido feito todo o trabalho de melhoramento.

    O objetivo principal deste trabalho é estabelecer um mecanismo de captura de perfis de usuários, através de uma eficiente mineração web de uso, de forma que a estrutura do site possa ser reorganizada de acordo com o perfil que o usuário que está acessando a página naquele momento se enquadre.

    Os sites de comércio eletrônico hoje em dia são muito acessados e possuem várias seções, essa grande quantidade de seções muitas vezes faz com que os usuários acabem se perdendo dentro do site, ou então acaba gerando um obstáculo para o objetivo final do site e do usuário, que é vender e comprar, respectivamente. A partir deste problema seria feita uma análise de padrões para que fosse medida a dificuldade que um usuário encontra na hora de efetuar uma compra.

\section{Metodologia}
    Para a realização deste trabalho primeiramente será realizada uma revisão bibliográfica a respeito da mineração web de uso; estudo aprofundado para decidir qual algoritmo de mineração utilizar para descoberta de perfis de usuários; analisar páginas de reclamações de usuários insafisfeitos e a partir daí aplicar a mineração de uso em algum site selecionado para descoberta dos perfis de usuários e então poder fazer uma remodelagem do layout desta página.

\section{Estrutura do trabalho}
    No Capítulo 2 será fundamentado os tipos de mineração web existentes, aprofundando a explanação na mineração web de uso. No capítulo 3 será mostrada a forma como será estruturado o modelo de mineração web de uso. No capítulo 4 serão explicados os métodos a as implementações considerando um caso de estudo; no capítulo 5 os resultados serão analisados considerando parâmetros de satisfação ou rejeição da página com usuários de teste. Por fim, no capítulo 6 serão apresentadas conclusões e formulações dos trabalhos futuros.

